\chapter{Inleiding}
In dit document beschrijven we hoe we een nieuwe versie van de ontwikkelde software gecontroleerd beschikbaar maken voor de eind gebruikers van de verschillende applicaties. We behandelen enkele onderwerpen zoals bijvoorbeeld welke omgevingen gebruikt worden, hoe de verschillende applicaties beschikbaar komen op deze omgevingen en wat de eisen aan de applicaties zijn voordat ze beschikbaar komen voor de eindgebruikers aan de hand van geautomatiseerde tests, gebruikerstests en code analyse.

\section{Bereiken van de verschillende applicaties}
De verschillende applicaties die ontwikkeld worden worden beschikbaar gesteld op infralab. Om met deze applicaties te verbinden, is een VPN verbinding nodig met vpninfralab.fhict.nl.

Wanneer deze verbinding is ingesteld, zijn de applicaties via bepaalde IP adressen te bereiken. De reeks IP adressen die voor onze applicaties gebruikt worden bevinden zich tussen 192.168.24.100-109.
Op dit moment is nog niet bekend welk IP adres de verschillende applicaties precies zullen gebruiken.

\section{Verklarende Woordenlijst}
Onderstaande tabel geeft een beschrijving van verschillende termen die in dit document gebruikt worden zodat de betekenis van deze termen duidelijker wordt.

\begin{center}
\begin{tabular}{ | l | p{10cm} |}
	\hline
	UI & User Interface, vaak een scherm of webpagina waardoor een applicatie gebruikt kan worden \\ \hline
	CLI & Command Line Interface, een command line of terminal venster waarmee een applicatie gebruikt kan worden \\ \hline
	SCRUM & Een methode om op een flexibele manier software te ontwikkelen \\ \hline
	CSRF & Cross Site Request Forgery, een manier om een aanval uit te voeren via een website \\ \hline
	SQL & Structured Query Language, een taal om verschillende commando's in een (SQL) database uit te voeren \\ \hline
	XSS & Cross Site Scripting, een manier om met behulp van javascript een aanval op een website uit te voeren \\
	\hline
\end{tabular}
\end{center}



