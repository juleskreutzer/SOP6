\chapter{Release Management}

\section{Content}
Elke nieuwe release bevat onderstaande gegevens:

\newlist{todolist1}{itemize}{2}
\setlist[todolist1]{label=$\square$}
\begin{itemize}
	\item Applicatie (.war bestand)
	\item Installatie handleiding
	\item Gebruikershandleiding (wanneer van toepassing)	
\end{itemize}

De verschillende versies van de applicatie zullen beschikbaar worden gesteld in een repository die bereikbaar is via \href{85.144.215.28:8082}{Artifactory}.

Voor elke release wordt tevens in Github een release gemaakt. In deze release op Github zijn de verschillende handleidingen en installatie instructies beschikbaar voor de corresponderende applicatie versie.

\section{Uitbrengen van een release}
Een nieuwe versie van de applicatie software wordt in de nacht van zondag op maandag uitgerold. Er is voor deze tijd gekozen omdat de verwachting is dat de gebruikers van de applicatie dan het minst actief zijn. De voorwaarden van het uitrollen van een nieuwe versie van de applicatie staat beschreven in hoofstuk 2.4 Voorwaarden van een release.

\section{Definition Of Done}
Wanneer een story of issue vanuit het backlog geïmplementeerd of verholpen is het de bedoeling dat de nieuwe code beschikbaar wordt op de productie omgeving. Voordat dit kan gebeuren, moet de code eerst getest en gecontroleerd/gereviewed worden. Een ontwikkelaar kan onderstaande Definition of Done gebruiken om te controleren of hij alle benodigde onderdelen heeft doorlopen
	
	\newlist{todolist2}{itemize}{2}
	\setlist[todolist2]{label=$\square$}
	\begin{itemize}
			\item Unit tests uitgevoerd?
			\item Code Review door andere ontwikkelaar
			\item Voorwaarden van een release behaald
			\item Code beschikbaar in test omgeving
			\item Code geaccepteerd door code review op github
			\item User Story/issue goedgekeurd door Product Owner		
	\end{itemize}

\section{Voorwaarden van een release}
Welke voorwaarden heeft een release voordat hij goedgekeurd wordt? Tests/Code analyse

\section{Review}
Wie reviewed een release? PO? Code review opzitten in github?

\section{Servers}
De verschillende applicaties die ontwikkeld worden zullen toegankelijk gemaakt worden op verschillende servers die voor ons worden aangeleverd op infralab.

\subsection{IP adres}

Om de kunnen verbinden met deze servers is een VPN verbinding met vpninfralab.fhict.nl vereist. De gebruikte IP adressen voor de verschillende applicaties zijn op dit moment nog niet vastgesteld. Wel is zeker dat elke applicatie een IP adres zal gebruiken van de IP adressen die voor ons beschikbaar zijn gemaakt in de reeks van 192.168.24.100-109.

\subsection{Applicatie server}
Voor het beschikbaar maken van de verschillende applicaties, is een Java EE applicatie server vereist. De applicaties die wij ontwikkelen zullen beschikbaar worden gemaakt en worden getest op de Glassfish applicatie server.