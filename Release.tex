\chapter{Release Management}
Dit hoofdstuk gaat over de verschillende stappen om te definieren wat een release is, wanneer deze in productie kan worden gezet en op welke servers deze software in productie wordt gezet.

\section{Content}
Welke content staat in een release? Software/manuals/instructies

\section{Uitbrengen van een release}
Wanneer wordt een release uitgebracht? Dag/tijdstip

\section{Definition Of Done}
Wanneer een story of issue vanuit het backlog geïmplementeerd of verholpen is het de bedoeling dat de nieuwe code beschikbaar wordt op de productie omgeving. Voordat dit kan gebeuren, moet de code eerst getest en gecontroleerd/gereviewed worden. Een ontwikkelaar kan onderstaande Definition of Done gebruiken om te controleren of hij alle benodigde onderdelen heeft doorlopen
	
	\newlist{todolist}{itemize}{2}
	\setlist[todolist]{label=$\square$}
	\begin{itemize}
			\item Unit tests uitgevoerd?
			\item Code Review door andere ontwikkelaar
			\item Voorwaarden van een release behaald
			\item Code beschikbaar in test omgeving
			\item Code geaccepteerd door code review op github
			\item User Story/issue goedgekeurd door Product Owner		
	\end{itemize}

\section{Voorwaarden van een release}
Welke voorwaarden heeft een release voordat hij goedgekeurd wordt? Tests/Code analyse

\section{Review}
Wie reviewed een release? PO? Code review opzitten in github?

\section{Servers}
De verschillende applicaties die ontwikkeld worden zullen toegankelijk gemaakt worden op verschillende servers die voor ons worden aangeleverd op infralab.

\subsection{IP adres}

Om de kunnen verbinden met deze servers is een VPN verbinding met vpninfralab.fhict.nl vereist. De gebruikte IP adressen voor de verschillende applicaties zijn op dit moment nog niet vastgesteld. Wel is zeker dat elke applicatie een IP adres zal gebruiken van de IP adressen die voor ons beschikbaar zijn gemaakt in de reeks van 192.168.24.100-109.

\subsection{Applicatie server}
Voor het beschikbaar maken van de verschillende applicaties, is een Java EE applicatie server vereist. De applicaties die wij ontwikkelen zullen beschikbaar worden gemaakt en worden getest op de Glassfish applicatie server.